\chapter*{Resumo}

% DO NOT CHANGE THIS - Add entry in the table of contents as chapter
\addcontentsline{toc}{chapter}{Resumo}



Sistemas capazes de reduzir o consumo de energia são necessários a fim de reduzir os custos energéticos para as empresas. Ao mesmo tempo, estudos mostram que a temperatura pode influenciar a produtividade humana, portanto, simplesmente remover sistemas HVAC poderia ter um impacto negativo na produtividade dos funcionários de uma empresa.

Hoje em dia \ac{BAS} são utilizados para gerir um edifício, no entanto sistemas tradicionais enfrentam um grande problema, eles têm custos elevados associados à instalação e equipamento. Ao mesmo tempo, o mercado é inundado com tablets baratos com sensores embutidos, conectividade e ecrãs.



Neste documento, propomos um sistema que utiliza um tablet Android, montado na parede da escritório, que executa uma aplicação para atingir automação do edifício. A nossa solução aborda o problema do consumo de energia, aumenta o conforto dos ocupantes e oferece segurança, incluindo a notificação de detecção de intrusos e monitorização de vídeo, para o utilizador. O sistema proposto difere dos sistemas centralizados tradicionais, oferecendo uma arquitetura distribuída com nós implantados em cada escritório e uma aplicação móvel de utilizador, capaz de interagir com o sistema.






\vspace{1cm}

% TODO - 4 a 6 palavras-chave;
\textbf{\Large Palavras-chave:} Android, Automação de edificios, Segurança, Detecção de presença, Sensores

\cleardoublepage

